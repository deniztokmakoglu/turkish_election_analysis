\documentclass[12pt]{article}
\usepackage{amsmath}
\usepackage{graphicx,psfrag,epsf}
\usepackage{enumerate}
\usepackage{natbib}
\usepackage{float}
\usepackage{booktabs,dcolumn}
\usepackage{indentfirst}
\usepackage{hyperref}
\usepackage{titlesec}
\usepackage{lscape}
\hypersetup{
    colorlinks=true,
    linkcolor=blue,
    filecolor=magenta,      
    urlcolor=cyan,
}
\newcolumntype{d}{D{.}{.}{2.5}}           % alignment on decimal marker
\newcommand\mc[1]{\multicolumn{1}{c}{#1}}
\usepackage{url} % not crucial - just used below for the URL 

%\pdfminorversion=4
% NOTE: To produce blinded version, replace "0" with "1" below.
\newcommand{\blind}{0}

% DON'T change margins - should be 1 inch all around.
\addtolength{\oddsidemargin}{-1in}%
\addtolength{\evensidemargin}{-1in}%
\addtolength{\textwidth}{1in}%
\addtolength{\textheight}{1.3in}%
\addtolength{\topmargin}{-.8in}%


\begin{document}

%\bibliographystyle{natbib}

\def\spacingset#1{\renewcommand{\baselinestretch}%
{#1}\small\normalsize} \spacingset{1}


%%%%%%%%%%%%%%%%%%%%%%%%%%%%%%%%%%%%%%%%%%%%%%%%%%%%%%%%%%%%%%%%%%%%%%%%%%%%%%

\if0\blind
{
  \title{\bf Turkish Elections and Catastrophic Events}
    {
}

  \maketitle
} \fi

\if1\blind
{
  \bigskip
  \bigskip
  \bigskip
  \begin{center}
    {\LARGE\bf Title}
\end{center}
  \medskip
} \fi

\spacingset{1.45} % DON'T change the spacing!
\section{Introduction}
\label{sec:intro}

A year ago, Turkey celebrated a century of being a democracy. Now, only 2 years away from the same milestone for being a republic, it is inevitable to think and analyse the trends of Turkish democracy. 
It is reasonable to say, Turkey, which resides possibly one of the lesser stable regions in the world; suffered a lot during this period, while trying to establish their democratic traditions. Having survived 3 coups, 3 "soft" coups and numerous economic, political and civil crises, Turkey still maintained regular and somewhat fair elections without interruption, unlike many other countries in a similar political landscape. The goal of this paper is to materialise this continuity and uncover the impact of catastrophic events in Turkey to voting behavior and results. 

\section{Background}

\subsection{Brief Summary of Turkish History}

After the end of World War I, like many empires in Europe; Ottoman Empire crumbled. In it's heartland, Turkey emerged as a new republic in 1923 after a long period of successive independence war against the colonial powers. 

Founded as a parliamentary republic, the first 17 years of the country has been defined by a single party period in which the founding party Republican People's Party, (Turkish abbreviation CHP will be used thereafter) was in power. The party is defined by having pro-European, Liberal and Secular values. Only in 1950, Turkey had it's truly first election with emergence of Right wing parties. Between 1950 - 2018, Turkey remained as a parliamentary republic with periodic municipal and parliamentary elections.

In 2018, with a 51\% referendum result, Turkey decided to transition into a presidential republic where the president and the Parliament is elected at the same time, using different ballots. For this project's scope Turkey's only 2 presidential elections were excluded from research. 

Inheriting many problems of it's predecessor, since it's foundation Turkey needed to deal with both regional and domestic problems. Composed of many different ethnic backgrounds such as Turkish, Kurdish, Armenian, Zaza etc. Turkey had to actively fight to maintain it's regional integrity. Moreover, as the gateway of Middle East to Europe, Turkey suffered from many regional events taking place outside of it's borders. One great example of this is the refugee influx to Turkey after ISIS war in Syria in 2011. 

\subsection{Summary of Catastrophic Events}

The research consists of 4 main catastrophic events, which will be used to analyze voter behaviour. 

\subsubsection{1999 Marmara Earthquake}



On August 17, 1999 the Marmara Region in Turkey was devastated from 





\section{Data and Limitations}

The data is retrieved retrieved from Turkish Statistical Institute, which is the main source for census and election data in Turkey. It consists municipal and parliamentary election results. Having decided to have a public vote for the president in 2010, Turkey had it's first presidential elections in 2014. Since there only has been one more presidential election after, this type of elections are not included in the analysis. 





\end{document}